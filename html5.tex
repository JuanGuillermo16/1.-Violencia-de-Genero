\documentclass[]{article}
\usepackage{lmodern}
\usepackage{amssymb,amsmath}
\usepackage{ifxetex,ifluatex}
\usepackage{fixltx2e} % provides \textsubscript
\ifnum 0\ifxetex 1\fi\ifluatex 1\fi=0 % if pdftex
  \usepackage[T1]{fontenc}
  \usepackage[utf8]{inputenc}
\else % if luatex or xelatex
  \ifxetex
    \usepackage{mathspec}
  \else
    \usepackage{fontspec}
  \fi
  \defaultfontfeatures{Ligatures=TeX,Scale=MatchLowercase}
\fi
% use upquote if available, for straight quotes in verbatim environments
\IfFileExists{upquote.sty}{\usepackage{upquote}}{}
% use microtype if available
\IfFileExists{microtype.sty}{%
\usepackage{microtype}
\UseMicrotypeSet[protrusion]{basicmath} % disable protrusion for tt fonts
}{}
\usepackage[margin=1in]{geometry}
\usepackage{hyperref}
\hypersetup{unicode=true,
            pdftitle={Predicción de Violencia Física Contra la Mujer en el Perú; Un enfoque de Big Data y Aprendizaje de Máquinas},
            pdfauthor={Nemías Seboya Ríos, Juan Guillermo Osío Jaramillo},
            pdfborder={0 0 0},
            breaklinks=true}
\urlstyle{same}  % don't use monospace font for urls
\usepackage{color}
\usepackage{fancyvrb}
\newcommand{\VerbBar}{|}
\newcommand{\VERB}{\Verb[commandchars=\\\{\}]}
\DefineVerbatimEnvironment{Highlighting}{Verbatim}{commandchars=\\\{\}}
% Add ',fontsize=\small' for more characters per line
\usepackage{framed}
\definecolor{shadecolor}{RGB}{248,248,248}
\newenvironment{Shaded}{\begin{snugshade}}{\end{snugshade}}
\newcommand{\KeywordTok}[1]{\textcolor[rgb]{0.13,0.29,0.53}{\textbf{#1}}}
\newcommand{\DataTypeTok}[1]{\textcolor[rgb]{0.13,0.29,0.53}{#1}}
\newcommand{\DecValTok}[1]{\textcolor[rgb]{0.00,0.00,0.81}{#1}}
\newcommand{\BaseNTok}[1]{\textcolor[rgb]{0.00,0.00,0.81}{#1}}
\newcommand{\FloatTok}[1]{\textcolor[rgb]{0.00,0.00,0.81}{#1}}
\newcommand{\ConstantTok}[1]{\textcolor[rgb]{0.00,0.00,0.00}{#1}}
\newcommand{\CharTok}[1]{\textcolor[rgb]{0.31,0.60,0.02}{#1}}
\newcommand{\SpecialCharTok}[1]{\textcolor[rgb]{0.00,0.00,0.00}{#1}}
\newcommand{\StringTok}[1]{\textcolor[rgb]{0.31,0.60,0.02}{#1}}
\newcommand{\VerbatimStringTok}[1]{\textcolor[rgb]{0.31,0.60,0.02}{#1}}
\newcommand{\SpecialStringTok}[1]{\textcolor[rgb]{0.31,0.60,0.02}{#1}}
\newcommand{\ImportTok}[1]{#1}
\newcommand{\CommentTok}[1]{\textcolor[rgb]{0.56,0.35,0.01}{\textit{#1}}}
\newcommand{\DocumentationTok}[1]{\textcolor[rgb]{0.56,0.35,0.01}{\textbf{\textit{#1}}}}
\newcommand{\AnnotationTok}[1]{\textcolor[rgb]{0.56,0.35,0.01}{\textbf{\textit{#1}}}}
\newcommand{\CommentVarTok}[1]{\textcolor[rgb]{0.56,0.35,0.01}{\textbf{\textit{#1}}}}
\newcommand{\OtherTok}[1]{\textcolor[rgb]{0.56,0.35,0.01}{#1}}
\newcommand{\FunctionTok}[1]{\textcolor[rgb]{0.00,0.00,0.00}{#1}}
\newcommand{\VariableTok}[1]{\textcolor[rgb]{0.00,0.00,0.00}{#1}}
\newcommand{\ControlFlowTok}[1]{\textcolor[rgb]{0.13,0.29,0.53}{\textbf{#1}}}
\newcommand{\OperatorTok}[1]{\textcolor[rgb]{0.81,0.36,0.00}{\textbf{#1}}}
\newcommand{\BuiltInTok}[1]{#1}
\newcommand{\ExtensionTok}[1]{#1}
\newcommand{\PreprocessorTok}[1]{\textcolor[rgb]{0.56,0.35,0.01}{\textit{#1}}}
\newcommand{\AttributeTok}[1]{\textcolor[rgb]{0.77,0.63,0.00}{#1}}
\newcommand{\RegionMarkerTok}[1]{#1}
\newcommand{\InformationTok}[1]{\textcolor[rgb]{0.56,0.35,0.01}{\textbf{\textit{#1}}}}
\newcommand{\WarningTok}[1]{\textcolor[rgb]{0.56,0.35,0.01}{\textbf{\textit{#1}}}}
\newcommand{\AlertTok}[1]{\textcolor[rgb]{0.94,0.16,0.16}{#1}}
\newcommand{\ErrorTok}[1]{\textcolor[rgb]{0.64,0.00,0.00}{\textbf{#1}}}
\newcommand{\NormalTok}[1]{#1}
\usepackage{longtable,booktabs}
\usepackage{graphicx,grffile}
\makeatletter
\def\maxwidth{\ifdim\Gin@nat@width>\linewidth\linewidth\else\Gin@nat@width\fi}
\def\maxheight{\ifdim\Gin@nat@height>\textheight\textheight\else\Gin@nat@height\fi}
\makeatother
% Scale images if necessary, so that they will not overflow the page
% margins by default, and it is still possible to overwrite the defaults
% using explicit options in \includegraphics[width, height, ...]{}
\setkeys{Gin}{width=\maxwidth,height=\maxheight,keepaspectratio}
\IfFileExists{parskip.sty}{%
\usepackage{parskip}
}{% else
\setlength{\parindent}{0pt}
\setlength{\parskip}{6pt plus 2pt minus 1pt}
}
\setlength{\emergencystretch}{3em}  % prevent overfull lines
\providecommand{\tightlist}{%
  \setlength{\itemsep}{0pt}\setlength{\parskip}{0pt}}
\setcounter{secnumdepth}{0}
% Redefines (sub)paragraphs to behave more like sections
\ifx\paragraph\undefined\else
\let\oldparagraph\paragraph
\renewcommand{\paragraph}[1]{\oldparagraph{#1}\mbox{}}
\fi
\ifx\subparagraph\undefined\else
\let\oldsubparagraph\subparagraph
\renewcommand{\subparagraph}[1]{\oldsubparagraph{#1}\mbox{}}
\fi

%%% Use protect on footnotes to avoid problems with footnotes in titles
\let\rmarkdownfootnote\footnote%
\def\footnote{\protect\rmarkdownfootnote}

%%% Change title format to be more compact
\usepackage{titling}

% Create subtitle command for use in maketitle
\providecommand{\subtitle}[1]{
  \posttitle{
    \begin{center}\large#1\end{center}
    }
}

\setlength{\droptitle}{-2em}

  \title{Predicción de Violencia Física Contra la Mujer en el Perú; Un enfoque de
Big Data y Aprendizaje de Máquinas}
    \pretitle{\vspace{\droptitle}\centering\huge}
  \posttitle{\par}
    \author{Nemías Seboya Ríos, Juan Guillermo Osío Jaramillo}
    \preauthor{\centering\large\emph}
  \postauthor{\par}
      \predate{\centering\large\emph}
  \postdate{\par}
    \date{23 de junio de 2019}

\usepackage{booktabs}
\usepackage{longtable}
\usepackage{array}
\usepackage{multirow}
\usepackage{wrapfig}
\usepackage{float}
\usepackage{colortbl}
\usepackage{pdflscape}
\usepackage{tabu}
\usepackage{threeparttable}
\usepackage{threeparttablex}
\usepackage[normalem]{ulem}
\usepackage{makecell}
\usepackage{xcolor}

\begin{document}
\maketitle

\paragraph{\texorpdfstring{\textbf{Motivación
General}}{Motivación General}}\label{motivacion-general}

\paragraph{La violencia contra la mujer es la que se ejerce por su
condición de mujer, siendo esta consecuencia de la discriminación que
sufre tanto en las leyes como en la práctica, y de la persistencia de
las desigualdades por razones de género. Esta violencia tiene un amplio
espectro de expresión, que incluye desde el menosprecio, hasta la
agresión física o psicológica y el asesinato. Entre 2010 y 2017, 837
mujeres fueron asesinadas y 1172 intentos de asesinato fueron llevados a
cabo en el Perú. El Instituto Nacional de Estadística e Informática de
dicho país emplea un concepto de violencia física cuya incidencia sobre
la población femenina se colocaba en 32.3\% para el año 2014. El
promedio de la misma estadística podría situarse en 35.92\% en los
últimos 10
años.}\label{la-violencia-contra-la-mujer-es-la-que-se-ejerce-por-su-condicion-de-mujer-siendo-esta-consecuencia-de-la-discriminacion-que-sufre-tanto-en-las-leyes-como-en-la-practica-y-de-la-persistencia-de-las-desigualdades-por-razones-de-genero.-esta-violencia-tiene-un-amplio-espectro-de-expresion-que-incluye-desde-el-menosprecio-hasta-la-agresion-fisica-o-psicologica-y-el-asesinato.-entre-2010-y-2017-837-mujeres-fueron-asesinadas-y-1172-intentos-de-asesinato-fueron-llevados-a-cabo-en-el-peru.-el-instituto-nacional-de-estadistica-e-informatica-de-dicho-pais-emplea-un-concepto-de-violencia-fisica-cuya-incidencia-sobre-la-poblacion-femenina-se-colocaba-en-32.3-para-el-ano-2014.-el-promedio-de-la-misma-estadistica-podria-situarse-en-35.92-en-los-ultimos-10-anos.}

\paragraph{Por otra parte, la entonces Ministra para la Mujer y
Poblaciones Vulnerables Ana María Mendieta expresaba en junio de 2018
que}\label{por-otra-parte-la-entonces-ministra-para-la-mujer-y-poblaciones-vulnerables-ana-maria-mendieta-expresaba-en-junio-de-2018-que}

``Las cifras nos demuestran que nuestra sociedad aún está influida por
el machismo, que impone ejercicio de poder por parte de los hombres
sobre las mujeres. Esta concepción machista es la que justifica y
reproduce la violencia familiar y sexual de la que somos testigos
diariamiente, y que constituye un riesgo para el bienestar y la vida de
las mujeres''

\paragraph{Un aspecto notable de la violencia contra la mujer es que
suele ajustarse a patrones sociales muy específicos y por lo tanto, su
ocurrencia suele estar acompañada tanto de factores de riesgo (edad,
pertenencia a minorías étnicas y clases económicas desfavorecidas, bajos
niveles de educación) como de informaciones preliminares que anuncian
una alta probabilidad del evento (historias de abuso familiar, ser
víctima de varias agresiones por parte de familiares y actores en el
entorno, etc.). La existencia de esta fuerte estructura de
determinaciones entre características individuales y que una mujer sea
víctima de violencia de género sugiere la factibilidad de elaborar una
estimación sobre la probabilidad de ocurrencia de este evento, lo cual
facilitaría la focalización de una política pública por medio de
identificación de poblaciones
vulnerables.}\label{un-aspecto-notable-de-la-violencia-contra-la-mujer-es-que-suele-ajustarse-a-patrones-sociales-muy-especificos-y-por-lo-tanto-su-ocurrencia-suele-estar-acompanada-tanto-de-factores-de-riesgo-edad-pertenencia-a-minorias-etnicas-y-clases-economicas-desfavorecidas-bajos-niveles-de-educacion-como-de-informaciones-preliminares-que-anuncian-una-alta-probabilidad-del-evento-historias-de-abuso-familiar-ser-victima-de-varias-agresiones-por-parte-de-familiares-y-actores-en-el-entorno-etc..-la-existencia-de-esta-fuerte-estructura-de-determinaciones-entre-caracteristicas-individuales-y-que-una-mujer-sea-victima-de-violencia-de-genero-sugiere-la-factibilidad-de-elaborar-una-estimacion-sobre-la-probabilidad-de-ocurrencia-de-este-evento-lo-cual-facilitaria-la-focalizacion-de-una-politica-publica-por-medio-de-identificacion-de-poblaciones-vulnerables.}

\paragraph{En este documento, reportamos los resultados de haber
prestado una asesoría técnica a un docente investigador que desarrolla
un sistema predictivo para la ocurrencia de violencia física contra la
mujer peruana. Describimos el nivel de desarrollo del proyecto que se
nos proporciona, nuestro diagnóstico y la recomendación e implementación
que aportamos al
proyecto.}\label{en-este-documento-reportamos-los-resultados-de-haber-prestado-una-asesoria-tecnica-a-un-docente-investigador-que-desarrolla-un-sistema-predictivo-para-la-ocurrencia-de-violencia-fisica-contra-la-mujer-peruana.-describimos-el-nivel-de-desarrollo-del-proyecto-que-se-nos-proporciona-nuestro-diagnostico-y-la-recomendacion-e-implementacion-que-aportamos-al-proyecto.}

\paragraph{\texorpdfstring{\textbf{1. Presentación del Material del
Proyecto}}{1. Presentación del Material del Proyecto}}\label{presentacion-del-material-del-proyecto}

\paragraph{\texorpdfstring{Para el propósito de esta asesoría nos fue
proporcionado un material de proyecto que abarca; (1) La base de datos
utilizada, (2) Un script de trabajo en formato de \emph{Jupiter
Notebook} que contiene el desarrollo de un sistema de predicción basado
en algoritmos de Machine Learning. A continuación, presentamos el
material
proporcionado:}{Para el propósito de esta asesoría nos fue proporcionado un material de proyecto que abarca; (1) La base de datos utilizada, (2) Un script de trabajo en formato de Jupiter Notebook que contiene el desarrollo de un sistema de predicción basado en algoritmos de Machine Learning. A continuación, presentamos el material proporcionado:}}\label{para-el-proposito-de-esta-asesoria-nos-fue-proporcionado-un-material-de-proyecto-que-abarca-1-la-base-de-datos-utilizada-2-un-script-de-trabajo-en-formato-de-jupiter-notebook-que-contiene-el-desarrollo-de-un-sistema-de-prediccion-basado-en-algoritmos-de-machine-learning.-a-continuacion-presentamos-el-material-proporcionado}

\paragraph{1.1 Presentación de los Datos
Proporcionados}\label{presentacion-de-los-datos-proporcionados}

\paragraph{\texorpdfstring{Nos fue otorgada una data-set basada en el
funcionamiento durante una más de década de una operación estadística
del \emph{Instituto Nacional de Estadística e Informática} del Perú. Por
lo tanto la base de datos que recibimos tiene un poco más 200000
registros de mujeres peruanas con la siguiente
información:}{Nos fue otorgada una data-set basada en el funcionamiento durante una más de década de una operación estadística del Instituto Nacional de Estadística e Informática del Perú. Por lo tanto la base de datos que recibimos tiene un poco más 200000 registros de mujeres peruanas con la siguiente información:}}\label{nos-fue-otorgada-una-data-set-basada-en-el-funcionamiento-durante-una-mas-de-decada-de-una-operacion-estadistica-del-instituto-nacional-de-estadistica-e-informatica-del-peru.-por-lo-tanto-la-base-de-datos-que-recibimos-tiene-un-poco-mas-200000-registros-de-mujeres-peruanas-con-la-siguiente-informacion}

\begin{itemize}
\item ~
  \paragraph{\texorpdfstring{Variables que describen el entorno material
  y cultural que habita el sujeto: Como la \textbf{región}, el
  \textbf{tipo de residencia} y el \textbf{tipo de lugar de
  crianza}}{Variables que describen el entorno material y cultural que habita el sujeto: Como la región, el tipo de residencia y el tipo de lugar de crianza}}\label{variables-que-describen-el-entorno-material-y-cultural-que-habita-el-sujeto-como-la-region-el-tipo-de-residencia-y-el-tipo-de-lugar-de-crianza}
\item ~
  \paragraph{\texorpdfstring{Variables que denotan el Perfil
  Sociodemográfico del individuo: como \textbf{grupo de edad}, la
  \textbf{etnicidad} y el \textbf{número de miembros del
  hogar}}{Variables que denotan el Perfil Sociodemográfico del individuo: como grupo de edad, la etnicidad y el número de miembros del hogar}}\label{variables-que-denotan-el-perfil-sociodemografico-del-individuo-como-grupo-de-edad-la-etnicidad-y-el-numero-de-miembros-del-hogar}
\item ~
  \paragraph{\texorpdfstring{Niveles de Educación del Individuo: En
  términos de \textbf{nivel educativo más alto alcanzado} y
  \textbf{grado de
  alfabetización}}{Niveles de Educación del Individuo: En términos de nivel educativo más alto alcanzado y grado de alfabetización}}\label{niveles-de-educacion-del-individuo-en-terminos-de-nivel-educativo-mas-alto-alcanzado-y-grado-de-alfabetizacion}
\item ~
  \paragraph{\texorpdfstring{Situación Económica del Individuo, medida
  por medio del \textbf{Índice de Riqueza} metodología
  INE.}{Situación Económica del Individuo, medida por medio del Índice de Riqueza metodología INE.}}\label{situacion-economica-del-individuo-medida-por-medio-del-indice-de-riqueza-metodologia-ine.}
\item ~
  \paragraph{\texorpdfstring{\textbf{Variables que informan sobre el
  maltrato físico por parte de diferentes actores en el entorno}: Como
  los padres, otros familiares y agentes externos a la
  familia.}{Variables que informan sobre el maltrato físico por parte de diferentes actores en el entorno: Como los padres, otros familiares y agentes externos a la familia.}}\label{variables-que-informan-sobre-el-maltrato-fisico-por-parte-de-diferentes-actores-en-el-entorno-como-los-padres-otros-familiares-y-agentes-externos-a-la-familia.}
\item ~
  \paragraph{\texorpdfstring{Una variable binaria que de identifica si
  la mujer \textbf{recibió o no maltrato
  físico}}{Una variable binaria que de identifica si la mujer recibió o no maltrato físico}}\label{una-variable-binaria-que-de-identifica-si-la-mujer-recibio-o-no-maltrato-fisico}
\end{itemize}

\paragraph{1.2 Análisis Exploratorio de
Variables}\label{analisis-exploratorio-de-variables}

\paragraph{(1) Tasa de Mujeres Afectadas por
Violencia}\label{tasa-de-mujeres-afectadas-por-violencia}

\paragraph{En primer lugar, presentamos nuestra estimación de la
cantidad de mujeres que han sido víctimas de violencia física, según el
concepto manejado por el
INEI}\label{en-primer-lugar-presentamos-nuestra-estimacion-de-la-cantidad-de-mujeres-que-han-sido-victimas-de-violencia-fisica-segun-el-concepto-manejado-por-el-inei}

\begin{verbatim}
## [1] "% de Mujeres que han recibido  Violencia Física: 35.92"
\end{verbatim}

\paragraph{\texorpdfstring{Es decir, que la probabilidad de que la mujer
típica peruana sea agredida se ha mantenido en niveles próximos al 36\%
durante los últimos 10 años. Esto es una estimación de probabilidad
incondicional, que todavía no incorpora el potencial de \emph{profiling}
representado por la base de
datos}{Es decir, que la probabilidad de que la mujer típica peruana sea agredida se ha mantenido en niveles próximos al 36\% durante los últimos 10 años. Esto es una estimación de probabilidad incondicional, que todavía no incorpora el potencial de profiling representado por la base de datos}}\label{es-decir-que-la-probabilidad-de-que-la-mujer-tipica-peruana-sea-agredida-se-ha-mantenido-en-niveles-proximos-al-36-durante-los-ultimos-10-anos.-esto-es-una-estimacion-de-probabilidad-incondicional-que-todavia-no-incorpora-el-potencial-de-profiling-representado-por-la-base-de-datos}

\paragraph{(2) Distribución por Grupos de
Edad}\label{distribucion-por-grupos-de-edad}

\paragraph{En cuanto a la distribución de la violencia por grupos de
edad, encontramos el patrón representado en la siguiente
gráfica:}\label{en-cuanto-a-la-distribucion-de-la-violencia-por-grupos-de-edad-encontramos-el-patron-representado-en-la-siguiente-grafica}

\begin{center}\includegraphics{html5_files/figure-latex/unnamed-chunk-4-1} \end{center}

\paragraph{Se aprecia un incremento sostenido de la tasa de violencia
cada grupo de edad. La interpretación de esto podía ser que el mayor
rechazo a las diferentes formas de violencia por parte de las nuevas
generaciones (efecto cultura) estaría sobrecompensando la indefensión
relativa de las mujeres comparativamente más jóvenes (efecto
vulnerabilidad). Otra explicación podría ser que las mujeres van
acumulando el riesgo de ser agredidas por lo menos una vez en la vida a
medida que transcurre su ciclo vitalicio; Es decir, a más años, más
experiencias que pudieron haber derivado en maltrato (por ejemplo,
relaciones con distintas parejas,
etc.).}\label{se-aprecia-un-incremento-sostenido-de-la-tasa-de-violencia-cada-grupo-de-edad.-la-interpretacion-de-esto-podia-ser-que-el-mayor-rechazo-a-las-diferentes-formas-de-violencia-por-parte-de-las-nuevas-generaciones-efecto-cultura-estaria-sobrecompensando-la-indefension-relativa-de-las-mujeres-comparativamente-mas-jovenes-efecto-vulnerabilidad.-otra-explicacion-podria-ser-que-las-mujeres-van-acumulando-el-riesgo-de-ser-agredidas-por-lo-menos-una-vez-en-la-vida-a-medida-que-transcurre-su-ciclo-vitalicio-es-decir-a-mas-anos-mas-experiencias-que-pudieron-haber-derivado-en-maltrato-por-ejemplo-relaciones-con-distintas-parejas-etc..}

\paragraph{(3) Distribución por
Departamentos}\label{distribucion-por-departamentos}

\paragraph{En cuanto a la distribución de la violencia por
departamentos, se pudo encontrar el patrón de agregación geográfica
representado a
continuación:}\label{en-cuanto-a-la-distribucion-de-la-violencia-por-departamentos-se-pudo-encontrar-el-patron-de-agregacion-geografica-representado-a-continuacion}

\begin{center}\includegraphics{html5_files/figure-latex/unnamed-chunk-5-1} \end{center}

\paragraph{Que permite corroborar la intuición de que la violencia está
concentrada geográficamente. Algunos departamentos presentan tasas de
violencia superiores al promedio nacional hasta en 10 puntos
porcentuales.}\label{que-permite-corroborar-la-intuicion-de-que-la-violencia-esta-concentrada-geograficamente.-algunos-departamentos-presentan-tasas-de-violencia-superiores-al-promedio-nacional-hasta-en-10-puntos-porcentuales.}

\paragraph{(4) Distribución por Niveles de
Riqueza}\label{distribucion-por-niveles-de-riqueza}

\paragraph{El INEI maneja una medida categórica de niveles de riqueza
basada en 5 niveles sucesivos, donde 5 representa el mejor perfil
económico del sujeto. Hallamos el siguiente patrón de la violencia por
niveles de
riqueza:}\label{el-inei-maneja-una-medida-categorica-de-niveles-de-riqueza-basada-en-5-niveles-sucesivos-donde-5-representa-el-mejor-perfil-economico-del-sujeto.-hallamos-el-siguiente-patron-de-la-violencia-por-niveles-de-riqueza}

\begin{center}\includegraphics{html5_files/figure-latex/unnamed-chunk-6-1} \end{center}

\paragraph{Interesantemente, la tasa de violencia no se comporta de
forma estrictamente decreciente con la riqueza de los individuos. De
hecho, la misma se incrementa hasta alcanzar un máximo global en el
tercer nivel de riqueza (39.41\%) para luego descender al mínimo global
en el último nivel de riqueza. Esto sugiere que solo un nivel de riqueza
extremadamente alto se convierte en un factor que contrarresta la
violencia de
género.}\label{interesantemente-la-tasa-de-violencia-no-se-comporta-de-forma-estrictamente-decreciente-con-la-riqueza-de-los-individuos.-de-hecho-la-misma-se-incrementa-hasta-alcanzar-un-maximo-global-en-el-tercer-nivel-de-riqueza-39.41-para-luego-descender-al-minimo-global-en-el-ultimo-nivel-de-riqueza.-esto-sugiere-que-solo-un-nivel-de-riqueza-extremadamente-alto-se-convierte-en-un-factor-que-contrarresta-la-violencia-de-genero.}

\paragraph{(5) Correlación entre las diferentes formas de
Violencia}\label{correlacion-entre-las-diferentes-formas-de-violencia}

\paragraph{Finalmente, proponemos la siguiente representación visual
(mapa de calor) de las correlaciones que existen entre diferentes formas
de
violencia.}\label{finalmente-proponemos-la-siguiente-representacion-visual-mapa-de-calor-de-las-correlaciones-que-existen-entre-diferentes-formas-de-violencia.}

\begin{center}\includegraphics{html5_files/figure-latex/unnamed-chunk-7-1} \end{center}

\paragraph{1.3 Presentación del Código
Proporcionado}\label{presentacion-del-codigo-proporcionado}

\paragraph{\texorpdfstring{Nuestro cliente desarrolló un sistema de
predicción basado en algoritmos de Machine Learning, pero desea mejorar
su funcionamiento a través del consejo de un experto. La solución
implementada por nuestro cliente se nos proporcionó en el formato de un
\textbf{\emph{jupyter notebook}} en lenguaje de programación
\textbf{Phyton}. El documento hace uso de las librerías de que
describimos a continuación (si no está interesado, saltar a nuestro
diagnóstico).}{Nuestro cliente desarrolló un sistema de predicción basado en algoritmos de Machine Learning, pero desea mejorar su funcionamiento a través del consejo de un experto. La solución implementada por nuestro cliente se nos proporcionó en el formato de un jupyter notebook en lenguaje de programación Phyton. El documento hace uso de las librerías de que describimos a continuación (si no está interesado, saltar a nuestro diagnóstico).}}\label{nuestro-cliente-desarrollo-un-sistema-de-prediccion-basado-en-algoritmos-de-machine-learning-pero-desea-mejorar-su-funcionamiento-a-traves-del-consejo-de-un-experto.-la-solucion-implementada-por-nuestro-cliente-se-nos-proporciono-en-el-formato-de-un-jupyter-notebook-en-lenguaje-de-programacion-phyton.-el-documento-hace-uso-de-las-librerias-de-que-describimos-a-continuacion-si-no-esta-interesado-saltar-a-nuestro-diagnostico.}

\paragraph{Uso de Librerías}\label{uso-de-librerias}

\begin{itemize}
\item ~
  \paragraph{\texorpdfstring{\textbf{numpy}:}{numpy:}}\label{numpy}
\item ~
  \paragraph{\texorpdfstring{\textbf{pandas}:}{pandas:}}\label{pandas}
\item ~
  \paragraph{\texorpdfstring{\textbf{matplotlib}:}{matplotlib:}}\label{matplotlib}
\item ~
  \paragraph{\texorpdfstring{\textbf{seaborn}:}{seaborn:}}\label{seaborn}
\item ~
  \paragraph{\texorpdfstring{\textbf{graphviz}:}{graphviz:}}\label{graphviz}
\item ~
  \paragraph{\texorpdfstring{\textbf{scikit
  learn}:}{scikit learn:}}\label{scikit-learn}
\end{itemize}

\paragraph{Workflow}\label{workflow}

\paragraph{A partir del código proporcionado, se identificò que la
solución implementada por el cliente correspondía a al siguiente diseño
del pipe-line de Machine-Learning. El código de transparencias de color
en la imagen prentende enfatizar el hecho de que el cliente omitiò
algunas instancias que nosotros consideramos reelevantes en el diseño
ideal/óptimo del pipeline de machine
learning.}\label{a-partir-del-codigo-proporcionado-se-identifico-que-la-solucion-implementada-por-el-cliente-correspondia-a-al-siguiente-diseno-del-pipe-line-de-machine-learning.-el-codigo-de-transparencias-de-color-en-la-imagen-prentende-enfatizar-el-hecho-de-que-el-cliente-omitio-algunas-instancias-que-nosotros-consideramos-reelevantes-en-el-diseno-idealoptimo-del-pipeline-de-machine-learning.}

\begin{figure}
\centering
\includegraphics{C:/Users/usuario/Desktop/Portafolio de Proyectos/1. Violencia de Género/imgs/1.3.1.png}
\caption{}
\end{figure}

\paragraph{\texorpdfstring{\textbf{2. Diagnóstico de la solución
Implementada:}}{2. Diagnóstico de la solución Implementada:}}\label{diagnostico-de-la-solucion-implementada}

\paragraph{Nosotros evaluamos el esquema de trabajo diseñado por nuestro
cliente, e identificamos los siguientes puntos en los que se puede
ofrecer un valor
agregado:}\label{nosotros-evaluamos-el-esquema-de-trabajo-disenado-por-nuestro-cliente-e-identificamos-los-siguientes-puntos-en-los-que-se-puede-ofrecer-un-valor-agregado}

\begin{itemize}
\item ~
  \paragraph{\texorpdfstring{\textbf{Feature Engineering}: En el
  ejercicio no hay una estrategia explícita de ingeniería de variables.
  Se asume que la mejor representación del sistema estudiado está en el
  espacio de los datos directamente observados (raw data). Se fuerza la
  representación numérica de las variables, pero esto genera que las
  variables nominales sin ordenamiento natural no aporten poder de
  predicción al modelo (e.g.~la región en la que se localiza la
  observación). Esto genera que el modelo desperdicie \textbf{insights
  predictivos} que provienen de la distribución geográfica de la
  violencia contra la mujer, y la incidencia de factores de riesgo como
  la etnicidad y el tipo de lugar de
  crianza.}{Feature Engineering: En el ejercicio no hay una estrategia explícita de ingeniería de variables. Se asume que la mejor representación del sistema estudiado está en el espacio de los datos directamente observados (raw data). Se fuerza la representación numérica de las variables, pero esto genera que las variables nominales sin ordenamiento natural no aporten poder de predicción al modelo (e.g.~la región en la que se localiza la observación). Esto genera que el modelo desperdicie insights predictivos que provienen de la distribución geográfica de la violencia contra la mujer, y la incidencia de factores de riesgo como la etnicidad y el tipo de lugar de crianza.}}\label{feature-engineering-en-el-ejercicio-no-hay-una-estrategia-explicita-de-ingenieria-de-variables.-se-asume-que-la-mejor-representacion-del-sistema-estudiado-esta-en-el-espacio-de-los-datos-directamente-observados-raw-data.-se-fuerza-la-representacion-numerica-de-las-variables-pero-esto-genera-que-las-variables-nominales-sin-ordenamiento-natural-no-aporten-poder-de-prediccion-al-modelo-e.g.la-region-en-la-que-se-localiza-la-observacion.-esto-genera-que-el-modelo-desperdicie-insights-predictivos-que-provienen-de-la-distribucion-geografica-de-la-violencia-contra-la-mujer-y-la-incidencia-de-factores-de-riesgo-como-la-etnicidad-y-el-tipo-de-lugar-de-crianza.}
\item ~
  \paragraph{\texorpdfstring{\textbf{Model Calibration:} El ejercicio
  carece de una estrategia explícita de calibración del modelo. Por lo
  tanto, no se garantiza que se identifique el modelo con el nivel de
  complejidad óptimo, con el mejor balance entre eficiencia estadística
  y precisión de la
  representación}{Model Calibration: El ejercicio carece de una estrategia explícita de calibración del modelo. Por lo tanto, no se garantiza que se identifique el modelo con el nivel de complejidad óptimo, con el mejor balance entre eficiencia estadística y precisión de la representación}}\label{model-calibration-el-ejercicio-carece-de-una-estrategia-explicita-de-calibracion-del-modelo.-por-lo-tanto-no-se-garantiza-que-se-identifique-el-modelo-con-el-nivel-de-complejidad-optimo-con-el-mejor-balance-entre-eficiencia-estadistica-y-precision-de-la-representacion}
\item ~
  \paragraph{\texorpdfstring{\textbf{Model Evaluation:} No se justifica
  la elección de la métrica de desempeño del modelo. Por lo tanto, la
  evaluación del modelo no está ponderando los costos de cada tipo de
  error. En este caso, un gran esfuerzo de modelación debería invertirse
  mejorar el \textbf{recall} del modelo sobre las etiquetas positivas.
  Esta medida representa la proporción de la población maltratada
  físicamente que es identificada por nuestro modelo. Esto permite
  generar una política pública con amplia cobertura sobre las mujeres
  maltratadas que mitigue los altos costos (en bienestar social) de la
  violencia de
  género.}{Model Evaluation: No se justifica la elección de la métrica de desempeño del modelo. Por lo tanto, la evaluación del modelo no está ponderando los costos de cada tipo de error. En este caso, un gran esfuerzo de modelación debería invertirse mejorar el recall del modelo sobre las etiquetas positivas. Esta medida representa la proporción de la población maltratada físicamente que es identificada por nuestro modelo. Esto permite generar una política pública con amplia cobertura sobre las mujeres maltratadas que mitigue los altos costos (en bienestar social) de la violencia de género.}}\label{model-evaluation-no-se-justifica-la-eleccion-de-la-metrica-de-desempeno-del-modelo.-por-lo-tanto-la-evaluacion-del-modelo-no-esta-ponderando-los-costos-de-cada-tipo-de-error.-en-este-caso-un-gran-esfuerzo-de-modelacion-deberia-invertirse-mejorar-el-recall-del-modelo-sobre-las-etiquetas-positivas.-esta-medida-representa-la-proporcion-de-la-poblacion-maltratada-fisicamente-que-es-identificada-por-nuestro-modelo.-esto-permite-generar-una-politica-publica-con-amplia-cobertura-sobre-las-mujeres-maltratadas-que-mitigue-los-altos-costos-en-bienestar-social-de-la-violencia-de-genero.}
\item ~
  \paragraph{\texorpdfstring{\textbf{Algorithms Performance:} Aunque se
  emplean algoritmos de high-performance, no se consideran las
  generalizaciones y extensiones de los mismos que típicamente obtienen
  mejor poder
  predictivo.}{Algorithms Performance: Aunque se emplean algoritmos de high-performance, no se consideran las generalizaciones y extensiones de los mismos que típicamente obtienen mejor poder predictivo.}}\label{algorithms-performance-aunque-se-emplean-algoritmos-de-high-performance-no-se-consideran-las-generalizaciones-y-extensiones-de-los-mismos-que-tipicamente-obtienen-mejor-poder-predictivo.}
\end{itemize}

\paragraph{2.2 Nuestro Plan de Trabajo:}\label{nuestro-plan-de-trabajo}

\paragraph{En base a las consideraciones planteadas en nuestro
diagnóstico anterior, presentamos nuestra plan de trabajo, como la
extensión del esquema del pipe-line de ML planteado por nuesto
cliente:}\label{en-base-a-las-consideraciones-planteadas-en-nuestro-diagnostico-anterior-presentamos-nuestra-plan-de-trabajo-como-la-extension-del-esquema-del-pipe-line-de-ml-planteado-por-nuesto-cliente}

\begin{figure}
\centering
\includegraphics{C:/Users/usuario/Desktop/Portafolio de Proyectos/1. Violencia de Género/imgs/2.4.1.png}
\caption{}
\end{figure}

\paragraph{En particular, nuestro plan de trabajo respeta en términos
generales la solución del cliente, pero propone las siguientes
innovaciones:}\label{en-particular-nuestro-plan-de-trabajo-respeta-en-terminos-generales-la-solucion-del-cliente-pero-propone-las-siguientes-innovaciones}

\begin{itemize}
\item ~
  \paragraph{\texorpdfstring{En términos de \textbf{ingeniería de
  variables}, se implementa la representación de
  \textbf{one-hot-enconding} de las variables categóricas sin
  ordenamiento natural. Esta estrategia consiste en que una variable
  categórica con una cantidad finita de K categorías se representa por K
  bits de información. Es decir, a cada observación se le asigna un
  vector k-dimensional de variables indicador, donde la única clase
  activa (que toma el valor de 1) es la que corresponde a la categoría a
  la que pertenece la observación. Esto se repite para todas las
  variables categóricas no ordinales. Como consecuencia de esta
  estrategia, el tamaño del \textbf{feature space} se incrementa
  considerablemente, lo cual no necesariamente deteriora el desempeño
  estadístico pues se emplean algoritmos especializados de minería de
  datos. En cambio, dicha representación si permite derivar poder
  discriminatorio de variables como la región, etnicidad y tipo de lugar
  de
  crianza.}{En términos de ingeniería de variables, se implementa la representación de one-hot-enconding de las variables categóricas sin ordenamiento natural. Esta estrategia consiste en que una variable categórica con una cantidad finita de K categorías se representa por K bits de información. Es decir, a cada observación se le asigna un vector k-dimensional de variables indicador, donde la única clase activa (que toma el valor de 1) es la que corresponde a la categoría a la que pertenece la observación. Esto se repite para todas las variables categóricas no ordinales. Como consecuencia de esta estrategia, el tamaño del feature space se incrementa considerablemente, lo cual no necesariamente deteriora el desempeño estadístico pues se emplean algoritmos especializados de minería de datos. En cambio, dicha representación si permite derivar poder discriminatorio de variables como la región, etnicidad y tipo de lugar de crianza.}}\label{en-terminos-de-ingenieria-de-variables-se-implementa-la-representacion-de-one-hot-enconding-de-las-variables-categoricas-sin-ordenamiento-natural.-esta-estrategia-consiste-en-que-una-variable-categorica-con-una-cantidad-finita-de-k-categorias-se-representa-por-k-bits-de-informacion.-es-decir-a-cada-observacion-se-le-asigna-un-vector-k-dimensional-de-variables-indicador-donde-la-unica-clase-activa-que-toma-el-valor-de-1-es-la-que-corresponde-a-la-categoria-a-la-que-pertenece-la-observacion.-esto-se-repite-para-todas-las-variables-categoricas-no-ordinales.-como-consecuencia-de-esta-estrategia-el-tamano-del-feature-space-se-incrementa-considerablemente-lo-cual-no-necesariamente-deteriora-el-desempeno-estadistico-pues-se-emplean-algoritmos-especializados-de-mineria-de-datos.-en-cambio-dicha-representacion-si-permite-derivar-poder-discriminatorio-de-variables-como-la-region-etnicidad-y-tipo-de-lugar-de-crianza.}
\item ~
  \paragraph{Por otra parte, el punto anterior si puede agravar el coste
  computacional (en tiempo y memoria), el cual escala rápidamente debido
  a la cantidad masiva de observaciones. Para relajar esta restricción,
  se propone incorporar el tiempo como predictor y restringir el
  entrenamiento a una ventana de estimación de cierto tamaño alrededor
  de una observación de referencia. Al hacer esto, se puede perder
  cierta eficiencia estadística pero también se gana precisión en la
  representación, pues observaciones que son más próximas temporalmente
  pueden ser más similares entre sí. Los esencial es que esto nos
  permite alcanzar un compromiso entre el tamaño de la representación y
  el tiempo de
  entrenamiento.}\label{por-otra-parte-el-punto-anterior-si-puede-agravar-el-coste-computacional-en-tiempo-y-memoria-el-cual-escala-rapidamente-debido-a-la-cantidad-masiva-de-observaciones.-para-relajar-esta-restriccion-se-propone-incorporar-el-tiempo-como-predictor-y-restringir-el-entrenamiento-a-una-ventana-de-estimacion-de-cierto-tamano-alrededor-de-una-observacion-de-referencia.-al-hacer-esto-se-puede-perder-cierta-eficiencia-estadistica-pero-tambien-se-gana-precision-en-la-representacion-pues-observaciones-que-son-mas-proximas-temporalmente-pueden-ser-mas-similares-entre-si.-los-esencial-es-que-esto-nos-permite-alcanzar-un-compromiso-entre-el-tamano-de-la-representacion-y-el-tiempo-de-entrenamiento.}
\item ~
  \paragraph{\texorpdfstring{En términos de \textbf{modelación}, se
  propone emplear la técnica de Gradient Boosting, un algoritmo de
  aprendizaje supervisado que se basa en construir de forma secuencial,
  un conjunto de modelos individualmente débiles. Esta estrategia
  empírica se explica con detalle en la sección de modelación del
  documento.}{En términos de modelación, se propone emplear la técnica de Gradient Boosting, un algoritmo de aprendizaje supervisado que se basa en construir de forma secuencial, un conjunto de modelos individualmente débiles. Esta estrategia empírica se explica con detalle en la sección de modelación del documento.}}\label{en-terminos-de-modelacion-se-propone-emplear-la-tecnica-de-gradient-boosting-un-algoritmo-de-aprendizaje-supervisado-que-se-basa-en-construir-de-forma-secuencial-un-conjunto-de-modelos-individualmente-debiles.-esta-estrategia-empirica-se-explica-con-detalle-en-la-seccion-de-modelacion-del-documento.}
\item ~
  \paragraph{\texorpdfstring{Para la \textbf{calibración} se proporciona
  al cliente una visualización del desempeño en el conjunto de
  evaluación de 50 modelos con diferentes asignaciones de parámetros. La
  visualización se genera sobre el espacio de \emph{recall vs
  precisión}, lo cual permite al \emph{policy-maker} racionalizar el
  desempeño de modelo como si fueran características de una política
  pública en etapa de
  diseño.}{Para la calibración se proporciona al cliente una visualización del desempeño en el conjunto de evaluación de 50 modelos con diferentes asignaciones de parámetros. La visualización se genera sobre el espacio de recall vs precisión, lo cual permite al policy-maker racionalizar el desempeño de modelo como si fueran características de una política pública en etapa de diseño.}}\label{para-la-calibracion-se-proporciona-al-cliente-una-visualizacion-del-desempeno-en-el-conjunto-de-evaluacion-de-50-modelos-con-diferentes-asignaciones-de-parametros.-la-visualizacion-se-genera-sobre-el-espacio-de-recall-vs-precision-lo-cual-permite-al-policy-maker-racionalizar-el-desempeno-de-modelo-como-si-fueran-caracteristicas-de-una-politica-publica-en-etapa-de-diseno.}
\end{itemize}

\paragraph{\texorpdfstring{\textbf{3. Desarrollo del Plan de Trabajo
}}{3. Desarrollo del Plan de Trabajo }}\label{desarrollo-del-plan-de-trabajo}

\paragraph{Puesto que el cliente nos entrega una base de datos limpia y
estandarizada, presentamos el desarrollo de nuesto plan de trabajopara
los pasos (2 a
5)}\label{puesto-que-el-cliente-nos-entrega-una-base-de-datos-limpia-y-estandarizada-presentamos-el-desarrollo-de-nuesto-plan-de-trabajopara-los-pasos-2-a-5}

\paragraph{3.2 Feature Engineering:}\label{feature-engineering}

\paragraph{El siguiente bloque de código contiene la transformación de
las variables categóricas que se justificó en el planteamiento del plan
de
trabajo.}\label{el-siguiente-bloque-de-codigo-contiene-la-transformacion-de-las-variables-categoricas-que-se-justifico-en-el-planteamiento-del-plan-de-trabajo.}

\begin{Shaded}
\begin{Highlighting}[]
\NormalTok{## (2) Block Wise Feature Enginering}

\NormalTok{num_var=}\KeywordTok{c}\NormalTok{(}
\NormalTok{  ## Mediciones numéricas y Categóricas Ordinales}
  
  \StringTok{"grupo_edad"}\NormalTok{,}
  \StringTok{"n_miembros_hogar"}\NormalTok{, }
  \StringTok{"educ"}\NormalTok{,}
  \StringTok{"nivel_alfa"}\NormalTok{, }
  \StringTok{"ind_riqueza"}\NormalTok{,}
  
\NormalTok{  ## Variables dicótomas que identifican diferentes fuentes}
\NormalTok{  ## de maltrato}
  
  \StringTok{"maltrato_padres"}\NormalTok{, }
  \StringTok{"maltrato_fam"}\NormalTok{,}
  \StringTok{"maltrato_padrastros"}\NormalTok{,}
  \StringTok{"matrato_otro"}\NormalTok{,    }
  \StringTok{"maltrato_padre_a_madre"}\NormalTok{)}


\NormalTok{## Porción de la información en var numéricas:}

\NormalTok{num_info=dplyr}\OperatorTok{::}\KeywordTok{select}\NormalTok{(ine_data, num_var )}


\NormalTok{## Aislar porción de la información en variables categóricas}

\NormalTok{non_num=}\KeywordTok{setdiff}\NormalTok{(}\KeywordTok{names}\NormalTok{(ine_data), num_var)}

\NormalTok{cat_var=}\KeywordTok{c}\NormalTok{(}\StringTok{"región",}
\StringTok{          "}\NormalTok{tipo_residencia}\StringTok{",}
\StringTok{          "}\NormalTok{tipo_lugar_crianza}\StringTok{",}
\StringTok{          "}\NormalTok{etnicidad}\StringTok{")}

\StringTok{cat_info=dplyr::select(ine_data, cat_var)}

\StringTok{head(cat_info)}
\end{Highlighting}
\end{Shaded}

\begin{verbatim}
##   región tipo_residencia tipo_lugar_crianza etnicidad
## 1      1               1                  0        10
## 2      1               1                  2        10
## 3      1               1                  2        10
## 4      1               1                  1        10
## 5      1               1                  2        10
## 6      1               1                  3        10
\end{verbatim}

\begin{Shaded}
\begin{Highlighting}[]
\NormalTok{## One hot Encoding Representation of Categorical Information}

\NormalTok{cat_info2=}\KeywordTok{lapply}\NormalTok{(cat_info, }\ControlFlowTok{function}\NormalTok{(x)\{ ans=dummies}\OperatorTok{::}\KeywordTok{dummy}\NormalTok{(x, }\DataTypeTok{sep=}\StringTok{"_"}\NormalTok{)\})}



\ControlFlowTok{for}\NormalTok{ (l }\ControlFlowTok{in} \DecValTok{1}\OperatorTok{:}\KeywordTok{length}\NormalTok{(cat_info2))\{}
  
  \KeywordTok{colnames}\NormalTok{(cat_info2[[l]])=}
\StringTok{    }
\StringTok{    }\KeywordTok{paste0}\NormalTok{(}\KeywordTok{names}\NormalTok{(cat_info2)[l], }\DecValTok{1}\OperatorTok{:}\KeywordTok{ncol}\NormalTok{(cat_info2[[l]])  )}
\NormalTok{\}}



\CommentTok{# lapply(cat_info, function(x)\{length(table(x))\})}
\CommentTok{# lapply(cat_info2, ncol)}
\CommentTok{# lapply(cat_info2, nrow)}

\CommentTok{# cat_info2=mapply(function, ...)}
\CommentTok{#   }
\NormalTok{## Reduce is one of Common Higher-Order Functions in Functional Programming Languages}

\NormalTok{cat_info3=}\KeywordTok{as.data.frame}\NormalTok{(}\KeywordTok{Reduce}\NormalTok{(}\DataTypeTok{f=}\ControlFlowTok{function}\NormalTok{(x,y)\{}\KeywordTok{cbind}\NormalTok{(x,y)\}, }\DataTypeTok{x=}\NormalTok{cat_info2))}

\NormalTok{## Construir tabla contacta con toda la información procesada}

\NormalTok{feature_matrix=}\KeywordTok{cbind}\NormalTok{(num_info, cat_info3)}

\NormalTok{## Diferencia en cantidad de Variables y tamaño de los objetos}

\KeywordTok{ncol}\NormalTok{(feature_matrix)}\OperatorTok{-}\KeywordTok{ncol}\NormalTok{(ine_data)}
\end{Highlighting}
\end{Shaded}

\begin{verbatim}
## [1] 38
\end{verbatim}

\begin{Shaded}
\begin{Highlighting}[]
\KeywordTok{object.size}\NormalTok{(feature_matrix)}\OperatorTok{-}\KeywordTok{object.size}\NormalTok{(ine_data)}
\end{Highlighting}
\end{Shaded}

\begin{verbatim}
## 31397992 bytes
\end{verbatim}

\paragraph{\texorpdfstring{La representación como
\emph{one-hot-encoding} consiste en 38 columnas adicionales que implican
31398064 bytes en el tamaño de la base de datos. La segunda
representación puede ser especialmente problemática para algoritmos que
hacen varios escaneos sobre la base de datos. El siguiente bloque de
código se refiere nuestra estrategia de retringir la ventana de
entrenamiento:}{La representación como one-hot-encoding consiste en 38 columnas adicionales que implican 31398064 bytes en el tamaño de la base de datos. La segunda representación puede ser especialmente problemática para algoritmos que hacen varios escaneos sobre la base de datos. El siguiente bloque de código se refiere nuestra estrategia de retringir la ventana de entrenamiento:}}\label{la-representacion-como-one-hot-encoding-consiste-en-38-columnas-adicionales-que-implican-31398064-bytes-en-el-tamano-de-la-base-de-datos.-la-segunda-representacion-puede-ser-especialmente-problematica-para-algoritmos-que-hacen-varios-escaneos-sobre-la-base-de-datos.-el-siguiente-bloque-de-codigo-se-refiere-nuestra-estrategia-de-retringir-la-ventana-de-entrenamiento}

\begin{Shaded}
\begin{Highlighting}[]
\NormalTok{### Conformar única base de datos con toda la información ya transformada:}

\CommentTok{# all_train_data=cbind(feature_matrix, train_lab=ine_data$lab_response)}
\CommentTok{# all_train_data$time_index=as.numeric(row.names(all_train_data))}
\CommentTok{# }
\CommentTok{# }
\CommentTok{# ## Escogemos el radio de la ventana de estimación, como proporción del tamaño de la muestra}
\CommentTok{# }
\CommentTok{# window_size=0.2}
\CommentTok{# }
\CommentTok{# ## Delimitamos a las observaciones que tienen suficiente información}
\CommentTok{# }
\CommentTok{# feasible_window=}
\CommentTok{#   round(c(}
\CommentTok{#     window_size*nrow(all_train_data)+1,}
\CommentTok{#     nrow(all_train_data)-window_size*nrow(all_train_data)}
\CommentTok{#   ))}
\CommentTok{# }
\CommentTok{# ## Escogemos de forma aleatoria una única observación dentro del conjunto}
\CommentTok{# ## factible}
\CommentTok{# }
\CommentTok{# }
\CommentTok{# eval_time=}
\CommentTok{#   sample(feasible_window[1]:feasible_window[2], size=1)}
\CommentTok{# }
\CommentTok{# ## Determinamos una ventana de estimación alrededor de esta observación}
\CommentTok{# }
\CommentTok{# eval_window= c(floor(eval_time-window_size*nrow(all_train_data)),}
\CommentTok{#                 floor(eval_time+window_size*nrow(all_train_data)))}
\CommentTok{# }
\CommentTok{# }
\CommentTok{# ## Restringimos los datos a una ventana de (window_size*100)% de la cantidad de observaciones antes y después de una observación de referencia}
\CommentTok{# }
\CommentTok{# effective_train_data=dplyr::filter(all_train_data, time_index %in%  eval_window[1]:eval_window[2])}
\CommentTok{# }
\CommentTok{# }
\CommentTok{# ## Corremos una lotería para identificar las Observaciones en Entrenamiento, Validación  y Prueba}
\CommentTok{# }
\CommentTok{# rows=1:nrow(effective_train_data)}
\CommentTok{# }
\CommentTok{# training_index=sample( rows , 0.4*nrow(effective_train_data))}
\CommentTok{# }
\CommentTok{# without_training= rows[!(rows %in% training_index)]}
\CommentTok{# }
\CommentTok{# validation_index=sample( without_training, 0.4*nrow(effective_train_data))}
\CommentTok{# }
\CommentTok{# test_index= rows[!(rows %in% training_index) & !(rows %in% validation_index) ]}
\CommentTok{# }
\CommentTok{# intersect(union(training_index, validation_index), test_index)}
\CommentTok{# }
\CommentTok{# }
\CommentTok{# ## Declarar data entrenamiento}
\CommentTok{# }
\CommentTok{# train_data0=effective_train_data[training_index, ]}
\CommentTok{# val_data0=effective_train_data[validation_index,]}
\CommentTok{# test_data0=effective_train_data[test_index, ]}
\CommentTok{# }
\CommentTok{# }
\CommentTok{# }
\CommentTok{# }
\CommentTok{# }
\CommentTok{# ## Area Plot with National Average}
\CommentTok{# }
\CommentTok{# all_train_data$in_sample=all_train_data$time_index %in% eval_window[1]:eval_window[2]}
\CommentTok{# }
\CommentTok{# }
\CommentTok{# library(ggplot2)}
\CommentTok{# }
\CommentTok{# setwd(root_wd);setwd("imgs")}
\CommentTok{# }
\CommentTok{# png(file="sample plot.png")}
\CommentTok{# ggplot(all_train_data, aes(x =time_index , y = time_index))+}
\CommentTok{# }
\CommentTok{#   ggtitle("")+}
\CommentTok{#   ylab("")+}
\CommentTok{#   xlab("")+}
\CommentTok{# }
\CommentTok{#   geom_line(col="firebrick", fill=alpha("firebrick", .3), size = 1)+}
\CommentTok{#   geom_vline(xintercept = feasible_window , col="black", lty=2)+}
\CommentTok{#   geom_vline(xintercept = eval_time, col="darkorchid", lty=2)+}
\CommentTok{# }
\CommentTok{#   geom_area(aes(x=time_index , y = time_index*in_sample),fill=alpha("darkorchid", .3) )+}
\CommentTok{# }
\CommentTok{# }
\CommentTok{#   theme_light();dev.off()}
\CommentTok{# }
\end{Highlighting}
\end{Shaded}

\paragraph{\texorpdfstring{Inicialmente se selecciona un parámetro
\(\alpha\) que sugiere la cantidad de información que va a ser empleada
en en todo el ejercicio (entrenamiento, calibración y prueba), expresada
como proporción del tamaño de la muestra. Posteriormente se identifican
las observaciones que poseen un margen de por lo menos
\(\frac{\alpha}{2}*N\) observaciones antes y después de dicha
observación. La observación de referencia (punteada morada) es
seleccionada de forma aleatoria de la ventana factible (entre las
punteadas negras). Alrededor de la observación de referencia, se
selecciona una ventana efectiva de estimación que contiene exactamente
un \(\frac{\alpha}{2}*N\) observaciones a cada lado de la referencia .
De esta forma, la cantidad de información para entrenamiento se reduce a
\(\alpha*N\) registros que son más similares, ya que están próximos en
el
tiempo.}{Inicialmente se selecciona un parámetro \textbackslash{}alpha que sugiere la cantidad de información que va a ser empleada en en todo el ejercicio (entrenamiento, calibración y prueba), expresada como proporción del tamaño de la muestra. Posteriormente se identifican las observaciones que poseen un margen de por lo menos \textbackslash{}frac\{\textbackslash{}alpha\}\{2\}*N observaciones antes y después de dicha observación. La observación de referencia (punteada morada) es seleccionada de forma aleatoria de la ventana factible (entre las punteadas negras). Alrededor de la observación de referencia, se selecciona una ventana efectiva de estimación que contiene exactamente un \textbackslash{}frac\{\textbackslash{}alpha\}\{2\}*N observaciones a cada lado de la referencia . De esta forma, la cantidad de información para entrenamiento se reduce a \textbackslash{}alpha*N registros que son más similares, ya que están próximos en el tiempo.}}\label{inicialmente-se-selecciona-un-parametro-alpha-que-sugiere-la-cantidad-de-informacion-que-va-a-ser-empleada-en-en-todo-el-ejercicio-entrenamiento-calibracion-y-prueba-expresada-como-proporcion-del-tamano-de-la-muestra.-posteriormente-se-identifican-las-observaciones-que-poseen-un-margen-de-por-lo-menos-fracalpha2n-observaciones-antes-y-despues-de-dicha-observacion.-la-observacion-de-referencia-punteada-morada-es-seleccionada-de-forma-aleatoria-de-la-ventana-factible-entre-las-punteadas-negras.-alrededor-de-la-observacion-de-referencia-se-selecciona-una-ventana-efectiva-de-estimacion-que-contiene-exactamente-un-fracalpha2n-observaciones-a-cada-lado-de-la-referencia-.-de-esta-forma-la-cantidad-de-informacion-para-entrenamiento-se-reduce-a-alphan-registros-que-son-mas-similares-ya-que-estan-proximos-en-el-tiempo.}

\paragraph{3.3 Modelling}\label{modelling}

\paragraph{(3.1) Representación}\label{representacion}

\paragraph{En términos de representación del sistema estudiado como
proceso estadístico, proponemos una implementación del algoritmo de
Gradient
Boosting.}\label{en-terminos-de-representacion-del-sistema-estudiado-como-proceso-estadistico-proponemos-una-implementacion-del-algoritmo-de-gradient-boosting.}

\paragraph{Gradient Boosting es una técnica de Machine Learning para
problemas de regresión y clasificación que genera un modelo predictivo a
través del ensamblaje de modelos individualmente débiles, típicamente
árboles de decisión. (2) Este construye el modelo global de forma
secuencial y se puede generalizar a diferentes funciones de
pérdida.}\label{gradient-boosting-es-una-tecnica-de-machine-learning-para-problemas-de-regresion-y-clasificacion-que-genera-un-modelo-predictivo-a-traves-del-ensamblaje-de-modelos-individualmente-debiles-tipicamente-arboles-de-decision.-2-este-construye-el-modelo-global-de-forma-secuencial-y-se-puede-generalizar-a-diferentes-funciones-de-perdida.}

\paragraph{(3.2) Estimación}\label{estimacion}

\paragraph{\texorpdfstring{En términos de estimación, podemos
interpretar el funcionamiento de los algoritmos de boosting como una
implementación de \emph{descenso de gradiente funcional}. Es decir, el
algoritmo trata de optimizar una función de costos sobre el conjunto de
modelos factibles al escoger la función (modelo) que que genera el mayor
descenso de
gradiente.}{En términos de estimación, podemos interpretar el funcionamiento de los algoritmos de boosting como una implementación de descenso de gradiente funcional. Es decir, el algoritmo trata de optimizar una función de costos sobre el conjunto de modelos factibles al escoger la función (modelo) que que genera el mayor descenso de gradiente.}}\label{en-terminos-de-estimacion-podemos-interpretar-el-funcionamiento-de-los-algoritmos-de-boosting-como-una-implementacion-de-descenso-de-gradiente-funcional.-es-decir-el-algoritmo-trata-de-optimizar-una-funcion-de-costos-sobre-el-conjunto-de-modelos-factibles-al-escoger-la-funcion-modelo-que-que-genera-el-mayor-descenso-de-gradiente.}

\paragraph{(3.3) Detalles técnicos}\label{detalles-tecnicos}

\paragraph{\texorpdfstring{A continuación, describimos algunos detalles
técnicos del algoritmo de \textbf{Gradient Boosting} según la
implementación concreta en la librería \textbf{gbm} de R
Studio.}{A continuación, describimos algunos detalles técnicos del algoritmo de Gradient Boosting según la implementación concreta en la librería gbm de R Studio.}}\label{a-continuacion-describimos-algunos-detalles-tecnicos-del-algoritmo-de-gradient-boosting-segun-la-implementacion-concreta-en-la-libreria-gbm-de-r-studio.}

\paragraph{\texorpdfstring{En cualquier problema de modelación
predictiva, deseamos encontrar una función de regresión \(\hat{f}(x)\)
que minimiza la expectativa de la función de pérdida
\(\psi(y, f)\)}{En cualquier problema de modelación predictiva, deseamos encontrar una función de regresión \textbackslash{}hat\{f\}(x) que minimiza la expectativa de la función de pérdida \textbackslash{}psi(y, f)}}\label{en-cualquier-problema-de-modelacion-predictiva-deseamos-encontrar-una-funcion-de-regresion-hatfx-que-minimiza-la-expectativa-de-la-funcion-de-perdida-psiy-f}

\paragraph{La operación de un algoritmo de Gradient Boosting depende de
los siguientes
parámetros:}\label{la-operacion-de-un-algoritmo-de-gradient-boosting-depende-de-los-siguientes-parametros}

\begin{itemize}
\item ~
  \paragraph{Función de pérdida}\label{funcion-de-perdida}
\item ~
  \paragraph{\texorpdfstring{Número de Iteraciones
  \(T\)}{Número de Iteraciones T}}\label{numero-de-iteraciones-t}
\item ~
  \paragraph{\texorpdfstring{Profundidad del árbol individual
  \(K\)}{Profundidad del árbol individual K}}\label{profundidad-del-arbol-individual-k}
\item ~
  \paragraph{\texorpdfstring{Tasa de Aprendizaje
  \(\lambda\)}{Tasa de Aprendizaje \textbackslash{}lambda}}\label{tasa-de-aprendizaje-lambda}
\item ~
  \paragraph{\texorpdfstring{Tasa SubMuestreo.
  \(p\)}{Tasa SubMuestreo. p}}\label{tasa-submuestreo.-p}
\end{itemize}

\begin{figure}
\centering
\includegraphics{C:/Users/usuario/Desktop/Portafolio de Proyectos/1. Violencia de Género/imgs/3.3.1.png}
\caption{}
\end{figure}

\begin{Shaded}
\begin{Highlighting}[]
\NormalTok{### (3) Modelling:}

\KeywordTok{library}\NormalTok{(randomForest)}



\NormalTok{## Definir Grilla de Hiper-Parámetros:}

\NormalTok{## (1) Para la Cantidad de Árboles}

\NormalTok{n_tree=}\KeywordTok{c}\NormalTok{(}\DecValTok{2000}\NormalTok{, }\DecValTok{4000}\NormalTok{, }\DecValTok{6000}\NormalTok{)}

\NormalTok{## (2) Profundidad de La Interacción}

\NormalTok{inter_depth=}\KeywordTok{seq}\NormalTok{(}\DataTypeTok{from=}\DecValTok{1}\NormalTok{, }\DataTypeTok{to=}\DecValTok{11}\NormalTok{, }\DataTypeTok{by=}\DecValTok{2}\NormalTok{)}

\NormalTok{## (3) Tamaño Mínimo del Nodo Terminal}

\NormalTok{node_size=}\KeywordTok{seq}\NormalTok{(}\DataTypeTok{from=}\DecValTok{15}\NormalTok{, }\DataTypeTok{to=}\DecValTok{30}\NormalTok{, }\DataTypeTok{by=}\DecValTok{2}\NormalTok{)}

\NormalTok{## Inicializar Tabla de Evaluación de Modelo}

\NormalTok{theta=}\KeywordTok{expand.grid}\NormalTok{(}\StringTok{"inter_depth"}\NormalTok{=inter_depth, }\StringTok{"min_node_size"}\NormalTok{=node_size)}

\NormalTok{theta}\OperatorTok{$}\NormalTok{presicion_si=}\OtherTok{NA}
\NormalTok{theta}\OperatorTok{$}\NormalTok{presicion_no=}\OtherTok{NA}

\NormalTok{theta}\OperatorTok{$}\NormalTok{recall_si=}\OtherTok{NA}
\NormalTok{theta}\OperatorTok{$}\NormalTok{recall_no=}\OtherTok{NA}

\NormalTok{## Loop de Calibración del Modelo}


\ControlFlowTok{for}\NormalTok{ (t }\ControlFlowTok{in} \DecValTok{1}\OperatorTok{:}\KeywordTok{nrow}\NormalTok{(theta))\{}
  
\NormalTok{  naive_boosting_fit=gbm}\OperatorTok{::}\KeywordTok{gbm}\NormalTok{(}
    
    \DataTypeTok{formula=}\KeywordTok{as.numeric}\NormalTok{(train_lab}\OperatorTok{==}\StringTok{"Sí"}\NormalTok{)}\OperatorTok{~}\NormalTok{. , ## response as function of all covariates,}
    
    \DataTypeTok{data=}\NormalTok{train_data0,}
    
    \DataTypeTok{distribution=}\StringTok{"bernoulli"}\NormalTok{, }
    
\NormalTok{    ## Complexity Parameters}
    
    \DataTypeTok{n.trees=} \DecValTok{2000}\NormalTok{, }
    \DataTypeTok{interaction.depth=}\NormalTok{ theta}\OperatorTok{$}\NormalTok{inter_depth[t],}
    \DataTypeTok{n.minobsinnode =}\NormalTok{ theta}\OperatorTok{$}\NormalTok{min_node_size[t]}
\NormalTok{  )}
  
  
  
  
\NormalTok{  ### Construyendo Métricas de Desempeño}
  
\NormalTok{  tab=}
\StringTok{    }\KeywordTok{table}\NormalTok{(}\StringTok{"ground_truth"}\NormalTok{=val_data0}\OperatorTok{$}\NormalTok{train_lab, ## Comparar con el conjunto de Evaluación}
          \StringTok{"predicted"}\NormalTok{=}
\StringTok{            }\NormalTok{( }\KeywordTok{ifelse}\NormalTok{(}\KeywordTok{predict}\NormalTok{(naive_boosting_fit, val_data0 , }\DataTypeTok{n.trees=}\DecValTok{2000}\NormalTok{,  }\DataTypeTok{type=}\StringTok{"response"}\NormalTok{)}\OperatorTok{>=}\FloatTok{0.5}\NormalTok{, }\StringTok{"Sí_pred"}\NormalTok{, }\StringTok{"No_pred"}\NormalTok{) )}
\NormalTok{    )}
  
  
\NormalTok{  ### Almacenar el desempeño del Modelo}
  
  
  
\NormalTok{  ## precision=}
  
\NormalTok{  theta}\OperatorTok{$}\NormalTok{presicion_si[t]=}
\StringTok{    }\NormalTok{tab[}\StringTok{"Sí"}\NormalTok{,}\StringTok{"Sí_pred"}\NormalTok{]}\OperatorTok{/}\NormalTok{(tab[}\StringTok{"Sí"}\NormalTok{,}\StringTok{"Sí_pred"}\NormalTok{]}\OperatorTok{+}\NormalTok{tab[}\StringTok{"No"}\NormalTok{,}\StringTok{"Sí_pred"}\NormalTok{])}
  
\NormalTok{  theta}\OperatorTok{$}\NormalTok{presicion_no[t]=}
\StringTok{    }\NormalTok{tab[}\StringTok{"No"}\NormalTok{,}\StringTok{"No_pred"}\NormalTok{]}\OperatorTok{/}\KeywordTok{sum}\NormalTok{(tab[,}\StringTok{"No_pred"}\NormalTok{])}
  
  
\NormalTok{  ## recall=}
\NormalTok{  theta}\OperatorTok{$}\NormalTok{recall_si[t]=}
\StringTok{    }\NormalTok{tab[}\StringTok{"Sí"}\NormalTok{,}\StringTok{"Sí_pred"}\NormalTok{]}\OperatorTok{/}\KeywordTok{sum}\NormalTok{(tab[}\StringTok{"Sí"}\NormalTok{,])}
  
\NormalTok{  theta}\OperatorTok{$}\NormalTok{recall_no[t]=}
\StringTok{    }\NormalTok{tab[}\StringTok{"No"}\NormalTok{,}\StringTok{"No_pred"}\NormalTok{]}\OperatorTok{/}\KeywordTok{sum}\NormalTok{(tab[}\StringTok{"No"}\NormalTok{,])}
  
  \KeywordTok{library}\NormalTok{(ggplot2)}
  
\NormalTok{  p=}
\StringTok{    }\KeywordTok{ggplot}\NormalTok{(}\DataTypeTok{data=}\NormalTok{theta, }\KeywordTok{aes}\NormalTok{(}\DataTypeTok{y=}\NormalTok{ presicion_si, }\DataTypeTok{x=}\NormalTok{recall_si))}\OperatorTok{+}
\StringTok{    }
\StringTok{    }\KeywordTok{ggtitle}\NormalTok{(}\StringTok{"Desempeño de los Modelos: Precisión vs Recall"}\NormalTok{)}\OperatorTok{+}
\StringTok{    }
\StringTok{    }\KeywordTok{geom_point}\NormalTok{(}\DataTypeTok{col=}\KeywordTok{alpha}\NormalTok{(}\StringTok{"darkorchid"}\NormalTok{, .}\DecValTok{3}\NormalTok{), }\DataTypeTok{size=}\DecValTok{5}\NormalTok{, }\DataTypeTok{lwd=}\DecValTok{2}\NormalTok{)}\OperatorTok{+}
\StringTok{    }\KeywordTok{theme_light}\NormalTok{()}
  
  
  \KeywordTok{setwd}\NormalTok{(root_wd)}
  \KeywordTok{setwd}\NormalTok{(}\StringTok{"results session"}\NormalTok{)}
  
  \KeywordTok{save}\NormalTok{(}\DataTypeTok{list=}\KeywordTok{c}\NormalTok{(}\StringTok{"theta"}\NormalTok{, }\StringTok{"p"}\NormalTok{), }
       \DataTypeTok{file=}\KeywordTok{sprintf}\NormalTok{(}\StringTok{"% s.RData"}\NormalTok{,}
\NormalTok{                    stringr}\OperatorTok{::}\KeywordTok{str_replace_all}\NormalTok{(}\KeywordTok{format}\NormalTok{(}\KeywordTok{Sys.time}\NormalTok{(), }\StringTok{"%a %b %d %X %Y"}\NormalTok{), }\DataTypeTok{pattern=}\StringTok{":"}\NormalTok{, }\DataTypeTok{replacement=}\StringTok{"_"}\NormalTok{)))}
  
\NormalTok{  ## Reportar Progreso del Algoritmo}
  
  \KeywordTok{print}\NormalTok{(}\KeywordTok{sprintf}\NormalTok{(}\StringTok{"Estimado el modelo %s de %s"}\NormalTok{, t, }\KeywordTok{nrow}\NormalTok{(theta)))}
\NormalTok{\}}
\end{Highlighting}
\end{Shaded}

\paragraph{4. Client Guided
Calibration}\label{client-guided-calibration}

\begin{center}\includegraphics{html5_files/figure-latex/unnamed-chunk-11-1} \end{center}

\begin{longtable}[]{@{}lrrrrrr@{}}
\toprule
& inter\_depth & min\_node\_size & presicion\_si & presicion\_no &
recall\_si & recall\_no\tabularnewline
\midrule
\endhead
1 & 1 & 15 & 55.7973 & 67.7257 & 24.0564 & 89.3188\tabularnewline
2 & 3 & 15 & 55.6500 & 68.1492 & 26.4716 & 88.1760\tabularnewline
3 & 5 & 15 & 54.7416 & 68.2026 & 27.3646 & 87.3198\tabularnewline
5 & 9 & 15 & 51.4656 & 68.3003 & 30.5718 & 83.8412\tabularnewline
6 & 11 & 15 & 51.0377 & 68.3885 & 31.4929 & 83.0668\tabularnewline
7 & 1 & 17 & 55.8372 & 67.8098 & 24.5001 & 89.1393\tabularnewline
8 & 3 & 17 & 54.6391 & 68.2545 & 27.7185 & 87.1026\tabularnewline
9 & 5 & 17 & 53.3237 & 68.3197 & 29.0609 & 85.7426\tabularnewline
10 & 7 & 17 & 52.5076 & 68.3544 & 29.9315 & 84.8265\tabularnewline
11 & 9 & 17 & 51.3696 & 68.1700 & 29.9146 & 84.1277\tabularnewline
12 & 11 & 17 & 50.7091 & 68.3018 & 31.3300 & 82.9314\tabularnewline
13 & 1 & 19 & 55.7491 & 67.7578 & 24.2642 & 89.2054\tabularnewline
14 & 3 & 19 & 54.7752 & 68.2337 & 27.5107 & 87.2694\tabularnewline
15 & 5 & 19 & 53.1421 & 68.2690 & 28.9261 & 85.7048\tabularnewline
16 & 7 & 19 & 52.7016 & 68.4006 & 30.0213 & 84.8989\tabularnewline
17 & 9 & 19 & 51.7004 & 68.2937 & 30.3134 & 84.1277\tabularnewline
20 & 3 & 21 & 54.3137 & 68.2192 & 27.7578 & 86.9137\tabularnewline
21 & 5 & 21 & 53.5529 & 68.3485 & 29.0384 & 85.8843\tabularnewline
22 & 7 & 21 & 52.7598 & 68.3690 & 29.7967 & 85.0469\tabularnewline
23 & 9 & 21 & 51.7555 & 68.3156 & 30.3864 & 84.1245\tabularnewline
24 & 11 & 21 & 51.0504 & 68.3729 & 31.3918 & 83.1298\tabularnewline
25 & 1 & 23 & 55.6121 & 67.7985 & 24.5731 & 89.0071\tabularnewline
26 & 3 & 23 & 54.7974 & 68.2211 & 27.4264 & 87.3198\tabularnewline
27 & 5 & 23 & 53.7235 & 68.4275 & 29.3361 & 85.8371\tabularnewline
28 & 7 & 23 & 52.2450 & 68.2556 & 29.6057 & 84.8328\tabularnewline
29 & 9 & 23 & 51.7962 & 68.3620 & 30.6111 & 84.0332\tabularnewline
30 & 11 & 23 & 51.1738 & 68.3432 & 31.0998 & 83.3690\tabularnewline
31 & 1 & 25 & 55.7560 & 67.8273 & 24.6461 & 89.0386\tabularnewline
32 & 3 & 25 & 54.8619 & 68.2323 & 27.4433 & 87.3450\tabularnewline
33 & 5 & 25 & 53.4034 & 68.3915 & 29.3923 & 85.6261\tabularnewline
34 & 7 & 25 & 52.6249 & 68.3969 & 30.0663 & 84.8297\tabularnewline
35 & 9 & 25 & 51.8261 & 68.3520 & 30.5268 & 84.0962\tabularnewline
36 & 11 & 25 & 51.3590 & 68.4497 & 31.5210 & 83.2683\tabularnewline
37 & 1 & 27 & 55.6232 & 67.8118 & 24.6405 & 88.9819\tabularnewline
38 & 3 & 27 & 54.8433 & 68.3183 & 27.9207 & 87.1152\tabularnewline
39 & 5 & 27 & 53.2238 & 68.3153 & 29.1171 & 85.6576\tabularnewline
40 & 7 & 27 & 52.5486 & 68.3618 & 29.9371 & 84.8486\tabularnewline
41 & 9 & 27 & 51.7769 & 68.2854 & 30.1955 & 84.2379\tabularnewline
42 & 11 & 27 & 51.3064 & 68.3859 & 31.2121 & 83.3973\tabularnewline
43 & 1 & 29 & 55.8005 & 67.7724 & 24.3148 & 89.2054\tabularnewline
44 & 3 & 29 & 54.9393 & 68.2910 & 27.7073 & 87.2631\tabularnewline
45 & 5 & 29 & 53.9335 & 68.4087 & 29.0721 & 86.0826\tabularnewline
46 & 7 & 29 & 52.4988 & 68.2868 & 29.5608 & 85.0091\tabularnewline
47 & 9 & 29 & 51.7532 & 68.3659 & 30.6729 & 83.9734\tabularnewline
48 & 11 & 29 & 51.3511 & 68.3480 & 30.9537 & 83.5642\tabularnewline
\bottomrule
\end{longtable}

\paragraph{\texorpdfstring{\textbf{5. Influencia de Variables y
evaluación FINAL
}}{5. Influencia de Variables y evaluación FINAL }}\label{influencia-de-variables-y-evaluacion-final}

\paragraph{\texorpdfstring{La siguiente medida de importancia de
variables está justificada en Friedman (2001) . Intuitivamente, la
medición está asociada a la contribución de la variable en términos de
poder clasificatorio del modelo, es decir, cuánto contribuye cada
variable a separar cada una de las clases. El poder clasificatorio de la
variable se mide en términos de el incremento en la pureza de los grupos
de predicción antes vs después de utilizar dicha variable. Es decir, se
estima que una variable contribuye más, en la medida que al generar una
partición de la muestra \emph{por esa variable}, se generan grupos
delimitados con resultados definidos (mujeres maltratadas vs no
maltratadas).}{La siguiente medida de importancia de variables está justificada en  Friedman (2001) . Intuitivamente, la medición está asociada a la contribución de la variable en términos de poder clasificatorio del modelo, es decir, cuánto contribuye cada variable a separar cada una de las clases. El poder clasificatorio de la variable se mide en términos de el incremento en la pureza de los grupos de predicción antes vs después de utilizar dicha variable. Es decir, se estima que una variable contribuye más, en la medida que al generar una partición de la muestra por esa variable, se generan grupos delimitados con resultados definidos (mujeres maltratadas vs no maltratadas).}}\label{la-siguiente-medida-de-importancia-de-variables-esta-justificada-en-friedman-2001-.-intuitivamente-la-medicion-esta-asociada-a-la-contribucion-de-la-variable-en-terminos-de-poder-clasificatorio-del-modelo-es-decir-cuanto-contribuye-cada-variable-a-separar-cada-una-de-las-clases.-el-poder-clasificatorio-de-la-variable-se-mide-en-terminos-de-el-incremento-en-la-pureza-de-los-grupos-de-prediccion-antes-vs-despues-de-utilizar-dicha-variable.-es-decir-se-estima-que-una-variable-contribuye-mas-en-la-medida-que-al-generar-una-particion-de-la-muestra-por-esa-variable-se-generan-grupos-delimitados-con-resultados-definidos-mujeres-maltratadas-vs-no-maltratadas.}

\begin{center}\includegraphics{html5_files/figure-latex/unnamed-chunk-13-1} \end{center}

\paragraph{Para interpretar estos resultados, por favor obsérvese que un
valor alto en importancia de variables implica que la variable es
importante para mejorar la clasificación entre mujeres maltratadas/no
maltratadas. Esto, con independencia de la dirección de la relación: Es
decir, la variable puede afectar positiva o negativamente la
probabilidad de maltrato. La dirección de la relación correspondiente se
puede rescatar por medio de intución e
interpretación.}\label{para-interpretar-estos-resultados-por-favor-observese-que-un-valor-alto-en-importancia-de-variables-implica-que-la-variable-es-importante-para-mejorar-la-clasificacion-entre-mujeres-maltratadasno-maltratadas.-esto-con-independencia-de-la-direccion-de-la-relacion-es-decir-la-variable-puede-afectar-positiva-o-negativamente-la-probabilidad-de-maltrato.-la-direccion-de-la-relacion-correspondiente-se-puede-rescatar-por-medio-de-intucion-e-interpretacion.}

\paragraph{\texorpdfstring{\textbf{6.
Referencias}}{6. Referencias}}\label{referencias}

\begin{longtable}[]{@{}l@{}}
\toprule
\begin{minipage}[b]{0.97\columnwidth}\raggedright\strut
References\strut
\end{minipage}\tabularnewline
\midrule
\endhead
\begin{minipage}[t]{0.97\columnwidth}\raggedright\strut
Ridgeway, G. (14 de Enero de 2019). Generalized Boosted Models: A guide
to the gbm package. Obtenido de
\url{http://127.0.0.1:21829/library/gbm/doc/gbm.pdf}\strut
\end{minipage}\tabularnewline
\begin{minipage}[t]{0.97\columnwidth}\raggedright\strut
Wikipedia. (s.f.). Feminicidios en el Perú. Obtenido de Wikipedia:
\url{https://es.wikipedia.org/wiki/Feminicidios_en_Per\%C3\%BA}\strut
\end{minipage}\tabularnewline
\begin{minipage}[t]{0.97\columnwidth}\raggedright\strut
Wikipedia. (s.f.). Gradient Boosting. Obtenido de Wikipedia:
\url{https://es.wikipedia.org/wiki/Gradient_boosting}\strut
\end{minipage}\tabularnewline
\begin{minipage}[t]{0.97\columnwidth}\raggedright\strut
Wikipedia. (s.f.). Violencia contra la Mujer. Obtenido de Wikipedia:
\url{https://es.wikipedia.org/wiki/Violencia_contra_la_mujer\#Per\%C3\%BA}\strut
\end{minipage}\tabularnewline
\bottomrule
\end{longtable}


\end{document}
